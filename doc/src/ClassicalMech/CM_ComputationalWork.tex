\documentclass[12pt, letterpaper, twoside]{report}
\usepackage[utf8]{inputenc}

\title{Classical Mechanics - Computational Work Documentaiton}
\author{Abudit P. Rai} \author{Dylan R. Smith}\\ Department of Physics and Astronomy, Michigan State University}
\date{Month Year}

\begin{document}

\begin{titlepage}
	\maketitle
\end{titlepage}

Following Professor Scott Pratt's lecture notes we thought week 1 work would be good for an introduciton of how to working with vectors and matrices in Python.\thanks{https://web.pa.msu.edu/people/pratts/phy321/lectures/lectures.pdf} Additionally, we thought it would be a good idea to show basic computational work through the classic canonball problem seen in introductory physics 1 courses. By having heavily outlined instructions in the Jupyter Notebook that show the students how to write algorithms in Python and then implement them into a funciton, this should be a fairly easy week for those wtihout any experience. 


It should be noted that this Week 1 notebook should also be given to students along with the general "Python Basics" notebook provided, as that notebook gives a better introduction for programing syntax, packages, and techniques. "CM Week1," on the other hand, provides a better introduction for computational work and the implementation of physics concepts into programing. 


\newpage
Week 2 work is a continuation of week 1's projectile motion, this time involving air resistance in the 2D canonball problem. We also involved a different form of information storage this time, using arrays to analyze our problem. Along with this, we introduced Euler's Method to the students. This is a simple yet effective tool in teaching about computational methods to students, from which we can dive into more complex tools in the future. 

We don't think it will be uncommon for there to be many students, especially those with no programing background, to find Week 2's work to be a sizable step above Week 1 and the Python Basics notebook. As this week does involve the concept of using an array to store data, the manipulation of such data types, and the first computational method that many will see.

\end{document}